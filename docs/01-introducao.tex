% Options for packages loaded elsewhere
\PassOptionsToPackage{unicode}{hyperref}
\PassOptionsToPackage{hyphens}{url}
\PassOptionsToPackage{dvipsnames,svgnames,x11names}{xcolor}
%
\documentclass[
]{article}

\usepackage{amsmath,amssymb}
\usepackage{iftex}
\ifPDFTeX
  \usepackage[T1]{fontenc}
  \usepackage[utf8]{inputenc}
  \usepackage{textcomp} % provide euro and other symbols
\else % if luatex or xetex
  \usepackage{unicode-math}
  \defaultfontfeatures{Scale=MatchLowercase}
  \defaultfontfeatures[\rmfamily]{Ligatures=TeX,Scale=1}
\fi
\usepackage{lmodern}
\ifPDFTeX\else  
    % xetex/luatex font selection
\fi
% Use upquote if available, for straight quotes in verbatim environments
\IfFileExists{upquote.sty}{\usepackage{upquote}}{}
\IfFileExists{microtype.sty}{% use microtype if available
  \usepackage[]{microtype}
  \UseMicrotypeSet[protrusion]{basicmath} % disable protrusion for tt fonts
}{}
\makeatletter
\@ifundefined{KOMAClassName}{% if non-KOMA class
  \IfFileExists{parskip.sty}{%
    \usepackage{parskip}
  }{% else
    \setlength{\parindent}{0pt}
    \setlength{\parskip}{6pt plus 2pt minus 1pt}}
}{% if KOMA class
  \KOMAoptions{parskip=half}}
\makeatother
\usepackage{xcolor}
\setlength{\emergencystretch}{3em} % prevent overfull lines
\setcounter{secnumdepth}{5}
% Make \paragraph and \subparagraph free-standing
\makeatletter
\ifx\paragraph\undefined\else
  \let\oldparagraph\paragraph
  \renewcommand{\paragraph}{
    \@ifstar
      \xxxParagraphStar
      \xxxParagraphNoStar
  }
  \newcommand{\xxxParagraphStar}[1]{\oldparagraph*{#1}\mbox{}}
  \newcommand{\xxxParagraphNoStar}[1]{\oldparagraph{#1}\mbox{}}
\fi
\ifx\subparagraph\undefined\else
  \let\oldsubparagraph\subparagraph
  \renewcommand{\subparagraph}{
    \@ifstar
      \xxxSubParagraphStar
      \xxxSubParagraphNoStar
  }
  \newcommand{\xxxSubParagraphStar}[1]{\oldsubparagraph*{#1}\mbox{}}
  \newcommand{\xxxSubParagraphNoStar}[1]{\oldsubparagraph{#1}\mbox{}}
\fi
\makeatother

\usepackage{color}
\usepackage{fancyvrb}
\newcommand{\VerbBar}{|}
\newcommand{\VERB}{\Verb[commandchars=\\\{\}]}
\DefineVerbatimEnvironment{Highlighting}{Verbatim}{commandchars=\\\{\}}
% Add ',fontsize=\small' for more characters per line
\usepackage{framed}
\definecolor{shadecolor}{RGB}{241,243,245}
\newenvironment{Shaded}{\begin{snugshade}}{\end{snugshade}}
\newcommand{\AlertTok}[1]{\textcolor[rgb]{0.68,0.00,0.00}{#1}}
\newcommand{\AnnotationTok}[1]{\textcolor[rgb]{0.37,0.37,0.37}{#1}}
\newcommand{\AttributeTok}[1]{\textcolor[rgb]{0.40,0.45,0.13}{#1}}
\newcommand{\BaseNTok}[1]{\textcolor[rgb]{0.68,0.00,0.00}{#1}}
\newcommand{\BuiltInTok}[1]{\textcolor[rgb]{0.00,0.23,0.31}{#1}}
\newcommand{\CharTok}[1]{\textcolor[rgb]{0.13,0.47,0.30}{#1}}
\newcommand{\CommentTok}[1]{\textcolor[rgb]{0.37,0.37,0.37}{#1}}
\newcommand{\CommentVarTok}[1]{\textcolor[rgb]{0.37,0.37,0.37}{\textit{#1}}}
\newcommand{\ConstantTok}[1]{\textcolor[rgb]{0.56,0.35,0.01}{#1}}
\newcommand{\ControlFlowTok}[1]{\textcolor[rgb]{0.00,0.23,0.31}{\textbf{#1}}}
\newcommand{\DataTypeTok}[1]{\textcolor[rgb]{0.68,0.00,0.00}{#1}}
\newcommand{\DecValTok}[1]{\textcolor[rgb]{0.68,0.00,0.00}{#1}}
\newcommand{\DocumentationTok}[1]{\textcolor[rgb]{0.37,0.37,0.37}{\textit{#1}}}
\newcommand{\ErrorTok}[1]{\textcolor[rgb]{0.68,0.00,0.00}{#1}}
\newcommand{\ExtensionTok}[1]{\textcolor[rgb]{0.00,0.23,0.31}{#1}}
\newcommand{\FloatTok}[1]{\textcolor[rgb]{0.68,0.00,0.00}{#1}}
\newcommand{\FunctionTok}[1]{\textcolor[rgb]{0.28,0.35,0.67}{#1}}
\newcommand{\ImportTok}[1]{\textcolor[rgb]{0.00,0.46,0.62}{#1}}
\newcommand{\InformationTok}[1]{\textcolor[rgb]{0.37,0.37,0.37}{#1}}
\newcommand{\KeywordTok}[1]{\textcolor[rgb]{0.00,0.23,0.31}{\textbf{#1}}}
\newcommand{\NormalTok}[1]{\textcolor[rgb]{0.00,0.23,0.31}{#1}}
\newcommand{\OperatorTok}[1]{\textcolor[rgb]{0.37,0.37,0.37}{#1}}
\newcommand{\OtherTok}[1]{\textcolor[rgb]{0.00,0.23,0.31}{#1}}
\newcommand{\PreprocessorTok}[1]{\textcolor[rgb]{0.68,0.00,0.00}{#1}}
\newcommand{\RegionMarkerTok}[1]{\textcolor[rgb]{0.00,0.23,0.31}{#1}}
\newcommand{\SpecialCharTok}[1]{\textcolor[rgb]{0.37,0.37,0.37}{#1}}
\newcommand{\SpecialStringTok}[1]{\textcolor[rgb]{0.13,0.47,0.30}{#1}}
\newcommand{\StringTok}[1]{\textcolor[rgb]{0.13,0.47,0.30}{#1}}
\newcommand{\VariableTok}[1]{\textcolor[rgb]{0.07,0.07,0.07}{#1}}
\newcommand{\VerbatimStringTok}[1]{\textcolor[rgb]{0.13,0.47,0.30}{#1}}
\newcommand{\WarningTok}[1]{\textcolor[rgb]{0.37,0.37,0.37}{\textit{#1}}}

\providecommand{\tightlist}{%
  \setlength{\itemsep}{0pt}\setlength{\parskip}{0pt}}\usepackage{longtable,booktabs,array}
\usepackage{calc} % for calculating minipage widths
% Correct order of tables after \paragraph or \subparagraph
\usepackage{etoolbox}
\makeatletter
\patchcmd\longtable{\par}{\if@noskipsec\mbox{}\fi\par}{}{}
\makeatother
% Allow footnotes in longtable head/foot
\IfFileExists{footnotehyper.sty}{\usepackage{footnotehyper}}{\usepackage{footnote}}
\makesavenoteenv{longtable}
\usepackage{graphicx}
\makeatletter
\newsavebox\pandoc@box
\newcommand*\pandocbounded[1]{% scales image to fit in text height/width
  \sbox\pandoc@box{#1}%
  \Gscale@div\@tempa{\textheight}{\dimexpr\ht\pandoc@box+\dp\pandoc@box\relax}%
  \Gscale@div\@tempb{\linewidth}{\wd\pandoc@box}%
  \ifdim\@tempb\p@<\@tempa\p@\let\@tempa\@tempb\fi% select the smaller of both
  \ifdim\@tempa\p@<\p@\scalebox{\@tempa}{\usebox\pandoc@box}%
  \else\usebox{\pandoc@box}%
  \fi%
}
% Set default figure placement to htbp
\def\fps@figure{htbp}
\makeatother
% definitions for citeproc citations
\NewDocumentCommand\citeproctext{}{}
\NewDocumentCommand\citeproc{mm}{%
  \begingroup\def\citeproctext{#2}\cite{#1}\endgroup}
\makeatletter
 % allow citations to break across lines
 \let\@cite@ofmt\@firstofone
 % avoid brackets around text for \cite:
 \def\@biblabel#1{}
 \def\@cite#1#2{{#1\if@tempswa , #2\fi}}
\makeatother
\newlength{\cslhangindent}
\setlength{\cslhangindent}{1.5em}
\newlength{\csllabelwidth}
\setlength{\csllabelwidth}{3em}
\newenvironment{CSLReferences}[2] % #1 hanging-indent, #2 entry-spacing
 {\begin{list}{}{%
  \setlength{\itemindent}{0pt}
  \setlength{\leftmargin}{0pt}
  \setlength{\parsep}{0pt}
  % turn on hanging indent if param 1 is 1
  \ifodd #1
   \setlength{\leftmargin}{\cslhangindent}
   \setlength{\itemindent}{-1\cslhangindent}
  \fi
  % set entry spacing
  \setlength{\itemsep}{#2\baselineskip}}}
 {\end{list}}
\usepackage{calc}
\newcommand{\CSLBlock}[1]{\hfill\break\parbox[t]{\linewidth}{\strut\ignorespaces#1\strut}}
\newcommand{\CSLLeftMargin}[1]{\parbox[t]{\csllabelwidth}{\strut#1\strut}}
\newcommand{\CSLRightInline}[1]{\parbox[t]{\linewidth - \csllabelwidth}{\strut#1\strut}}
\newcommand{\CSLIndent}[1]{\hspace{\cslhangindent}#1}

\usepackage{booktabs}
\usepackage{longtable}
\usepackage{array}
\usepackage{multirow}
\usepackage{wrapfig}
\usepackage{float}
\usepackage{colortbl}
\usepackage{pdflscape}
\usepackage{tabu}
\usepackage{threeparttable}
\usepackage{threeparttablex}
\usepackage[normalem]{ulem}
\usepackage{makecell}
\usepackage{xcolor}
\usepackage{fontspec}
\usepackage{multicol}
\usepackage{hhline}
\newlength\Oldarrayrulewidth
\newlength\Oldtabcolsep
\usepackage{hyperref}
\makeatletter
\@ifpackageloaded{caption}{}{\usepackage{caption}}
\AtBeginDocument{%
\ifdefined\contentsname
  \renewcommand*\contentsname{Índice}
\else
  \newcommand\contentsname{Índice}
\fi
\ifdefined\listfigurename
  \renewcommand*\listfigurename{Lista de Figuras}
\else
  \newcommand\listfigurename{Lista de Figuras}
\fi
\ifdefined\listtablename
  \renewcommand*\listtablename{Lista de Tabelas}
\else
  \newcommand\listtablename{Lista de Tabelas}
\fi
\ifdefined\figurename
  \renewcommand*\figurename{Figura}
\else
  \newcommand\figurename{Figura}
\fi
\ifdefined\tablename
  \renewcommand*\tablename{Tabela}
\else
  \newcommand\tablename{Tabela}
\fi
}
\@ifpackageloaded{float}{}{\usepackage{float}}
\floatstyle{ruled}
\@ifundefined{c@chapter}{\newfloat{codelisting}{h}{lop}}{\newfloat{codelisting}{h}{lop}[chapter]}
\floatname{codelisting}{Listagem}
\newcommand*\listoflistings{\listof{codelisting}{Lista de Listagens}}
\makeatother
\makeatletter
\makeatother
\makeatletter
\@ifpackageloaded{caption}{}{\usepackage{caption}}
\@ifpackageloaded{subcaption}{}{\usepackage{subcaption}}
\makeatother

\ifLuaTeX
\usepackage[bidi=basic]{babel}
\else
\usepackage[bidi=default]{babel}
\fi
\babelprovide[main,import]{portuguese}
% get rid of language-specific shorthands (see #6817):
\let\LanguageShortHands\languageshorthands
\def\languageshorthands#1{}
\usepackage{bookmark}

\IfFileExists{xurl.sty}{\usepackage{xurl}}{} % add URL line breaks if available
\urlstyle{same} % disable monospaced font for URLs
\hypersetup{
  pdftitle={Desigualdades nos acessos às Ciências Sociais:},
  pdfauthor={Artur Damião; Murillo Marschner Alves de Brito; Isabella da Nóbrega; Gabriella Medeiros; Maria Eduarda; Yasmin Flor},
  pdflang={pt},
  pdfkeywords={Estratificação educacional, Ensino superior, Capital
cultural},
  colorlinks=true,
  linkcolor={blue},
  filecolor={Maroon},
  citecolor={Blue},
  urlcolor={Blue},
  pdfcreator={LaTeX via pandoc}}


\title{Desigualdades nos acessos às Ciências Sociais:}
\usepackage{etoolbox}
\makeatletter
\providecommand{\subtitle}[1]{% add subtitle to \maketitle
  \apptocmd{\@title}{\par {\large #1 \par}}{}{}
}
\makeatother
\subtitle{O papel da raça, renda e escolaridade na aprovação do
vestibular}
\author{Artur Damião \and Murillo Marschner Alves de Brito \and Isabella
da Nóbrega \and Gabriella Medeiros \and Maria Eduarda \and Yasmin Flor}
\date{}

\begin{document}
\maketitle

\renewcommand*\contentsname{Índice}
{
\hypersetup{linkcolor=}
\setcounter{tocdepth}{3}
\tableofcontents
}

\subsection{Introdução}\label{introduuxe7uxe3o}

De acordo com Fernandes et al. (2020), etc.

\subsection{Metodologia}\label{metodologia}

A partir da base de dados da FUVEST, referente aos anos de 2019 a 2023.
Ao todo, somam-se 4088 observações distribuídas entre os anos.

Com isso em mente, estabelecemos, portanto, a seguinte pergunta de
pesquisa: \textbf{como a renda, raça e o tipo de escola inlfuenciam o
ingresso de alunos cotistas e não cotistas na carreira de Ciências
Sociais da Universidade de São Paulo}? Para as variáveis analisadas,
estabelecemos:

\begin{longtable}[]{@{}
  >{\raggedright\arraybackslash}p{(\linewidth - 4\tabcolsep) * \real{0.3472}}
  >{\raggedright\arraybackslash}p{(\linewidth - 4\tabcolsep) * \real{0.3333}}
  >{\raggedright\arraybackslash}p{(\linewidth - 4\tabcolsep) * \real{0.3194}}@{}}
\toprule\noalign{}
\begin{minipage}[b]{\linewidth}\raggedright
Variável
\end{minipage} & \begin{minipage}[b]{\linewidth}\raggedright
Tipo
\end{minipage} & \begin{minipage}[b]{\linewidth}\raggedright
Categoria/Descrição
\end{minipage} \\
\midrule\noalign{}
\endhead
\bottomrule\noalign{}
\endlastfoot
\textbf{Convocado para matrícula (VD)} & Binária & 1 = Convocado para
matricular-se na carreira de Ciências Sociais pela FUVEST 0 = Não
convocado \\
\textbf{Raça} & Categórica (dummy) & 0 = Branco (referência) 1 =
Preto/Pardo/Indígena \\
\textbf{Renda familiar} & Contínua & Faixas de renda em salários
mínimos \\
\textbf{Escolaridade dos pais} & Ordinal & 1 = Baixa (até ensino
fundamental completo) 2 = Média (ensino médio completo) 3 = Alta (ensino
superior completo ou mais) \\
\textbf{Tipo de Ensino Médio} & Categórica (dummy) & 0 = Escola pública
(referência) 1 = Escola privada \\
\end{longtable}

Para compreender o impacto de cada uma das variáveis, estabelecemos o
seguinte modelo de regressão logística:

\[
\log \left(\frac{P(Aprovado=1)}{1 - P(Aprovado=1)}\right) = \beta_0 + \beta_1 \text{Raça} + \beta_2 \text{Renda} + \beta_3 \text{Escolaridade} + \beta_4 \text{TipoEscola} + \varepsilon
\]

\subsection{Hipóteses de pesquisa}\label{hipuxf3teses-de-pesquisa}

Frente ao exposto, consideramos as seguintes hipóteses:

\textbf{H\textsubscript{0}}: Raça, renda e tipo de escola não
influenciam o ingresso de estudantes na carreira de Ciências Sociais.
\textbf{H\textsubscript{1}}: A renda familiar e a escolaridade dos pais
(proxy para capital cultural) aumentam as chances de ingresso (analisar
dados da FUVEST, comparando os inscritos na carreira X perfil presente
no curso, via questionário PET).\\
\textbf{H\textsubscript{2}}: Alunos oriundos de escolas privadas têm
maior probabilidade de ingressar sem necessidade de cotas, mesmo em
grupos de baixa renda.

No contexto da análise das hipóteses propostas, o objetivo é investigar
como variáveis sociodemográficas, como raça, renda familiar e tipo de
escola, influenciam o ingresso de estudantes na carreira de Ciências
Sociais. A hipótese nula (\textbf{H\textsubscript{0}}) sugere que essas
variáveis não têm impacto significativo sobre a aprovação no vestibular,
implicando que fatores como a origem socioeconômica dos candidatos não
determinariam as suas chances de ingresso. Para testar essa hipótese,
compararemos as características dos estudantes inscritos na FUVEST para
a carreira de Ciências Sociais com o perfil dos alunos já aprovados, a
partir dos dados do questionário PET.

Por outro lado, a hipótese alternativa (\textbf{H\textsubscript{1}})
propõe que a renda familiar e a escolaridade dos pais (como proxy para o
capital cultural) têm um efeito positivo nas chances de ingresso,
favorecendo candidatos com maior capital cultural. A relação entre essas
variáveis e o ingresso será analisada por meio do modelo de regressão
apresentado, verificando a consistência e a força dessa associação nos
dados coletados. Comparando o perfil dos inscritos com o dos aprovados,
será possível avaliar a influência desses fatores no processo seletivo e
nas desigualdades de acesso à educação superior.

\subsubsection{Modelo de Regressão
Logística}\label{modelo-de-regressuxe3o-loguxedstica}

\begin{longtable}[]{@{}cccc@{}}
\toprule\noalign{}
\endhead
\bottomrule\noalign{}
\endlastfoot
~ & \multicolumn{3}{c@{}}{%
matricula} \\
Predictors & Risk Ratios & CI & p \\
(Intercept) & 0.22 & 0.18~--~0.25 & \textbf{\textless0.001} \\
renda & 1.13 & 1.10~--~1.16 & \textbf{\textless0.001} \\
escolaridade {[}linear{]} & 0.94 & 0.84~--~1.06 & 0.328 \\
escolaridade {[}quadratic{]} & 1.10 & 0.99~--~1.21 & 0.064 \\
raca & 0.91 & 0.81~--~1.02 & 0.098 \\
ensino med & 1.00 & 0.95~--~1.05 & 0.979 \\
cursinho & 1.08 & 0.98~--~1.18 & 0.112 \\
Observations & \multicolumn{3}{l@{}}{%
4088} \\
R\textsuperscript{2} Nagelkerke & \multicolumn{3}{l@{}}{%
0.063} \\
\end{longtable}

Frequência da matrícula:

\begin{longtable}[]{@{}lrr@{}}
\caption{Distribuição da variável `matricula'}\tabularnewline
\toprule\noalign{}
Matricula & Frequência & Proporção \\
\midrule\noalign{}
\endfirsthead
\toprule\noalign{}
Matricula & Frequência & Proporção \\
\midrule\noalign{}
\endhead
\bottomrule\noalign{}
\endlastfoot
0 & 3367 & 0.82 \\
1 & 721 & 0.18 \\
\end{longtable}

\begin{verbatim}

    Hosmer and Lemeshow goodness of fit (GOF) test

data:  mod$y, fitted(mod)
X-squared = 4.5904, df = 8, p-value = 0.8003
\end{verbatim}

\begin{longtable}[]{@{}rrl@{}}
\caption{Teste Omnibus para os coeficientes do modelo}\tabularnewline
\toprule\noalign{}
Qui-quadrado & GL & Valor-p \\
\midrule\noalign{}
\endfirsthead
\toprule\noalign{}
Qui-quadrado & GL & Valor-p \\
\midrule\noalign{}
\endhead
\bottomrule\noalign{}
\endlastfoot
159.8 & 6 & p \textless{} 0,001 \\
\end{longtable}

\begin{longtable}[]{@{}
  >{\raggedright\arraybackslash}p{(\linewidth - 12\tabcolsep) * \real{0.2000}}
  >{\raggedleft\arraybackslash}p{(\linewidth - 12\tabcolsep) * \real{0.0933}}
  >{\raggedleft\arraybackslash}p{(\linewidth - 12\tabcolsep) * \real{0.1600}}
  >{\raggedleft\arraybackslash}p{(\linewidth - 12\tabcolsep) * \real{0.1067}}
  >{\raggedleft\arraybackslash}p{(\linewidth - 12\tabcolsep) * \real{0.0800}}
  >{\raggedleft\arraybackslash}p{(\linewidth - 12\tabcolsep) * \real{0.1333}}
  >{\raggedleft\arraybackslash}p{(\linewidth - 12\tabcolsep) * \real{0.2267}}@{}}
\caption{Coeficientes do modelo}\tabularnewline
\toprule\noalign{}
\begin{minipage}[b]{\linewidth}\raggedright
\end{minipage} & \begin{minipage}[b]{\linewidth}\raggedleft
B
\end{minipage} & \begin{minipage}[b]{\linewidth}\raggedleft
Erro Padrão
\end{minipage} & \begin{minipage}[b]{\linewidth}\raggedleft
Z
\end{minipage} & \begin{minipage}[b]{\linewidth}\raggedleft
Sig.
\end{minipage} & \begin{minipage}[b]{\linewidth}\raggedleft
exp(B)
\end{minipage} & \begin{minipage}[b]{\linewidth}\raggedleft
(exp(B)-1) x 100
\end{minipage} \\
\midrule\noalign{}
\endfirsthead
\toprule\noalign{}
\begin{minipage}[b]{\linewidth}\raggedright
\end{minipage} & \begin{minipage}[b]{\linewidth}\raggedleft
B
\end{minipage} & \begin{minipage}[b]{\linewidth}\raggedleft
Erro Padrão
\end{minipage} & \begin{minipage}[b]{\linewidth}\raggedleft
Z
\end{minipage} & \begin{minipage}[b]{\linewidth}\raggedleft
Sig.
\end{minipage} & \begin{minipage}[b]{\linewidth}\raggedleft
exp(B)
\end{minipage} & \begin{minipage}[b]{\linewidth}\raggedleft
(exp(B)-1) x 100
\end{minipage} \\
\midrule\noalign{}
\endhead
\bottomrule\noalign{}
\endlastfoot
(Intercept) & -1.536 & 0.083 & -18.397 & 0.000 & 0.2152327 &
-78.4767348 \\
renda & 0.122 & 0.012 & 10.159 & 0.000 & 1.1293740 & 12.9374010 \\
escolaridade.L & -0.059 & 0.061 & -0.978 & 0.328 & 0.9424087 &
-5.7591282 \\
escolaridade.Q & 0.094 & 0.050 & 1.854 & 0.064 & 1.0980530 &
9.8052968 \\
raca & -0.097 & 0.058 & -1.655 & 0.098 & 0.9078922 & -9.2107790 \\
ensino\_med & 0.001 & 0.025 & 0.026 & 0.979 & 1.0006636 & 0.0663554 \\
cursinho & 0.075 & 0.047 & 1.588 & 0.112 & 1.0779144 & 7.7914385 \\
\end{longtable}

\begin{verbatim}
'log Lik.' 3648.94 (df=7)
\end{verbatim}

\begin{verbatim}
'log Lik.' 3808.74 (df=1)
\end{verbatim}

\begin{verbatim}
McFadden 
   0.042 
\end{verbatim}

\begin{verbatim}
CoxSnell 
   0.038 
\end{verbatim}

\begin{verbatim}
Nagelkerke 
     0.063 
\end{verbatim}

\begin{verbatim}
[1] 3707.151
\end{verbatim}

\begin{verbatim}
[1] 3817.057
\end{verbatim}

\begin{verbatim}
                    Medida Valor.Modelo.Logit Valor.Modelo.Nulo
1 -2 Log-Likelihood (-2LL)           3648.940          3808.740
2    Pseudo R² de McFadden              0.042                NA
3 Pseudo R² de Cox & Snell              0.038                NA
4  Pseudo R² de Nagelkerke              0.063                NA
5                      BIC           3707.151          3817.057
\end{verbatim}

\global\setlength{\Oldarrayrulewidth}{\arrayrulewidth}

\global\setlength{\Oldtabcolsep}{\tabcolsep}

\setlength{\tabcolsep}{2pt}

\renewcommand*{\arraystretch}{1.5}



\providecommand{\ascline}[3]{\noalign{\global\arrayrulewidth #1}\arrayrulecolor[HTML]{#2}\cline{#3}}

\begin{longtable*}[c]{|p{0.75in}|p{0.75in}|p{0.75in}}



\ascline{1.5pt}{666666}{1-3}

\multicolumn{1}{>{\raggedright}m{\dimexpr 0.75in+0\tabcolsep}}{\textcolor[HTML]{000000}{\fontsize{11}{11}\selectfont{\global\setmainfont{Arial}{Medida}}}} & \multicolumn{1}{>{\raggedleft}m{\dimexpr 0.75in+0\tabcolsep}}{\textcolor[HTML]{000000}{\fontsize{11}{11}\selectfont{\global\setmainfont{Arial}{Valor.Modelo.Logit}}}} & \multicolumn{1}{>{\raggedleft}m{\dimexpr 0.75in+0\tabcolsep}}{\textcolor[HTML]{000000}{\fontsize{11}{11}\selectfont{\global\setmainfont{Arial}{Valor.Modelo.Nulo}}}} \\

\ascline{1.5pt}{666666}{1-3}\endfirsthead 

\ascline{1.5pt}{666666}{1-3}

\multicolumn{1}{>{\raggedright}m{\dimexpr 0.75in+0\tabcolsep}}{\textcolor[HTML]{000000}{\fontsize{11}{11}\selectfont{\global\setmainfont{Arial}{Medida}}}} & \multicolumn{1}{>{\raggedleft}m{\dimexpr 0.75in+0\tabcolsep}}{\textcolor[HTML]{000000}{\fontsize{11}{11}\selectfont{\global\setmainfont{Arial}{Valor.Modelo.Logit}}}} & \multicolumn{1}{>{\raggedleft}m{\dimexpr 0.75in+0\tabcolsep}}{\textcolor[HTML]{000000}{\fontsize{11}{11}\selectfont{\global\setmainfont{Arial}{Valor.Modelo.Nulo}}}} \\

\ascline{1.5pt}{666666}{1-3}\endhead



\multicolumn{1}{>{\raggedright}m{\dimexpr 0.75in+0\tabcolsep}}{\textcolor[HTML]{000000}{\fontsize{11}{11}\selectfont{\global\setmainfont{Arial}{-2\ Log-Likelihood\ (-2LL)}}}} & \multicolumn{1}{>{\raggedleft}m{\dimexpr 0.75in+0\tabcolsep}}{\textcolor[HTML]{000000}{\fontsize{11}{11}\selectfont{\global\setmainfont{Arial}{3,648.940}}}} & \multicolumn{1}{>{\raggedleft}m{\dimexpr 0.75in+0\tabcolsep}}{\textcolor[HTML]{000000}{\fontsize{11}{11}\selectfont{\global\setmainfont{Arial}{3,808.740}}}} \\





\multicolumn{1}{>{\raggedright}m{\dimexpr 0.75in+0\tabcolsep}}{\textcolor[HTML]{000000}{\fontsize{11}{11}\selectfont{\global\setmainfont{Arial}{Pseudo\ R²\ de\ McFadden}}}} & \multicolumn{1}{>{\raggedleft}m{\dimexpr 0.75in+0\tabcolsep}}{\textcolor[HTML]{000000}{\fontsize{11}{11}\selectfont{\global\setmainfont{Arial}{0.042}}}} & \multicolumn{1}{>{\raggedleft}m{\dimexpr 0.75in+0\tabcolsep}}{\textcolor[HTML]{000000}{\fontsize{11}{11}\selectfont{\global\setmainfont{Arial}{}}}} \\





\multicolumn{1}{>{\raggedright}m{\dimexpr 0.75in+0\tabcolsep}}{\textcolor[HTML]{000000}{\fontsize{11}{11}\selectfont{\global\setmainfont{Arial}{Pseudo\ R²\ de\ Cox\ \&\ Snell}}}} & \multicolumn{1}{>{\raggedleft}m{\dimexpr 0.75in+0\tabcolsep}}{\textcolor[HTML]{000000}{\fontsize{11}{11}\selectfont{\global\setmainfont{Arial}{0.038}}}} & \multicolumn{1}{>{\raggedleft}m{\dimexpr 0.75in+0\tabcolsep}}{\textcolor[HTML]{000000}{\fontsize{11}{11}\selectfont{\global\setmainfont{Arial}{}}}} \\





\multicolumn{1}{>{\raggedright}m{\dimexpr 0.75in+0\tabcolsep}}{\textcolor[HTML]{000000}{\fontsize{11}{11}\selectfont{\global\setmainfont{Arial}{Pseudo\ R²\ de\ Nagelkerke}}}} & \multicolumn{1}{>{\raggedleft}m{\dimexpr 0.75in+0\tabcolsep}}{\textcolor[HTML]{000000}{\fontsize{11}{11}\selectfont{\global\setmainfont{Arial}{0.063}}}} & \multicolumn{1}{>{\raggedleft}m{\dimexpr 0.75in+0\tabcolsep}}{\textcolor[HTML]{000000}{\fontsize{11}{11}\selectfont{\global\setmainfont{Arial}{}}}} \\





\multicolumn{1}{>{\raggedright}m{\dimexpr 0.75in+0\tabcolsep}}{\textcolor[HTML]{000000}{\fontsize{11}{11}\selectfont{\global\setmainfont{Arial}{BIC}}}} & \multicolumn{1}{>{\raggedleft}m{\dimexpr 0.75in+0\tabcolsep}}{\textcolor[HTML]{000000}{\fontsize{11}{11}\selectfont{\global\setmainfont{Arial}{3,707.151}}}} & \multicolumn{1}{>{\raggedleft}m{\dimexpr 0.75in+0\tabcolsep}}{\textcolor[HTML]{000000}{\fontsize{11}{11}\selectfont{\global\setmainfont{Arial}{3,817.057}}}} \\

\ascline{1.5pt}{666666}{1-3}



\end{longtable*}



\arrayrulecolor[HTML]{000000}

\global\setlength{\arrayrulewidth}{\Oldarrayrulewidth}

\global\setlength{\tabcolsep}{\Oldtabcolsep}

\renewcommand*{\arraystretch}{1}

\begin{Shaded}
\begin{Highlighting}[]
\CommentTok{\# calculando os valores preditos do modelo}
\NormalTok{pred }\OtherTok{\textless{}{-}} \FunctionTok{predict}\NormalTok{(mod, }\AttributeTok{type =} \StringTok{\textquotesingle{}response\textquotesingle{}}\NormalTok{)}

\CommentTok{\# OBS: Calcula o percentual das margens (total), indicando que probabilidades }
\CommentTok{\#     preditas maiores que 0.5 serão consideradas 1 (ou seja, acerto)}

\FunctionTok{addmargins}\NormalTok{(}\FunctionTok{prop.table}\NormalTok{(}\FunctionTok{table}\NormalTok{(mod}\SpecialCharTok{$}\NormalTok{y, pred }\SpecialCharTok{\textgreater{}} \FloatTok{0.3}\NormalTok{)))}
\end{Highlighting}
\end{Shaded}

\begin{verbatim}
     
           FALSE       TRUE        Sum
  0   0.75684932 0.06678082 0.82363014
  1   0.14456947 0.03180039 0.17636986
  Sum 0.90141879 0.09858121 1.00000000
\end{verbatim}

\begin{Shaded}
\begin{Highlighting}[]
\CommentTok{\# Matriz de confusão com corte de 0.3}
\FunctionTok{table}\NormalTok{(mod}\SpecialCharTok{$}\NormalTok{y, pred }\SpecialCharTok{\textgreater{}} \FloatTok{0.3}\NormalTok{)}
\end{Highlighting}
\end{Shaded}

\begin{verbatim}
   
    FALSE TRUE
  0  3094  273
  1   591  130
\end{verbatim}

\phantomsection\label{refs}
\begin{CSLReferences}{1}{0}
\bibitem[\citeproctext]{ref-fernandes2020}
Fernandes, Antônio Alves Tôrres, Dalson Britto Figueiredo Filho,
Enivaldo Carvalho Da Rocha, e Willber Da Silva Nascimento. 2020. {«Read
this paper if you want to learn logistic regression»}. \emph{Revista de
Sociologia e Política} 28 (74): 006.
\url{https://doi.org/10.1590/1678-987320287406en}.

\end{CSLReferences}




\end{document}
